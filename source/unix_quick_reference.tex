\documentclass[a4paper]{article}
    \usepackage{fullpage}
    \usepackage{multicol}
    \usepackage{pdflscape}
    \usepackage{textcomp}
    \usepackage[utf8]{inputenc}
    \usepackage[T1]{fontenc}
    \usepackage{geometry}
    \usepackage{array}
    \usepackage{varwidth}
    \textheight=10in
    \pagestyle{empty}
    \raggedright

\def\bull{\vrule height 0.8ex width .7ex depth -.1ex }
\geometry{a4paper, 
total={200mm,257mm},
 left=20mm,
 top=10mm,
 bottom=10mm
 }

 \begin{document}
 \begin{landscape}
 \setlength\columnsep{5pt}
 \begin{multicols}{3}

%==== Title  ====%
\begin{center}
	{\large \textbf{Quick guide to SSH/SFTP/Unix}}\\
	\textit{Author: Taffie Coler}\\
	\textit{taffie.coler@gmail.com}\\
\end{center}

%==== Column 1 ====%
Notice: These are very basic, and not all commands nor are all the users of the commands listed here. This is for a beginner's quick reference, and has some tips and tricks for starting out.

\vspace{2mm}
\textbf{What is SSH?}\\
Secure Shell is a protocol used to remotely access your
shell account, proxy, etc. securely over a network.\\

\vspace{2mm}
\textbf{Connecting using SFTP or SSH}\\
sftp [username]@[remote-path]\\
ex: sftp slayerx@p1.cs.ohiou.edu\\
ssh [username]@[remote-path]\\
ex: ssh slayerx@p2.cs.ohiou.edu\\

\vspace{2mm}
\textbf{Wildcard Characters}\\
The \textit{asterisk} * represents any amount of characters.
g++ *.cc says, “g++, compile all of my .cc files please”
The \textit{question mark} ? represents exactly one character.
ls file.??? says, “list all files with a 3 character extension.\\

\vspace{2mm}
\textbf{Terminology}\\
\textit{remote path/r-path} – path of remote computer\\
\textit{local path/l-path} – path of computer you are on\\
Bracketed words – are in place of strings, do not include\\
brackets when you type commands in.\\

\vspace{2mm}
\textbf{Redirecting standard input and output}\\
The \textit{pipe operator} | feeds the output of one command to
the input of another. For example, \verb+history | grep ssh+ will
 output any line in your command history where
the string ssh occurs.\\
The \textit{output redirection operator} > will direct the output of
a command into a file. If the file does not exist, it will
create it, if it does, it will \textbf{overwrite} the old file.\\
\textit{Another output redirection operator} >> directs the output
of a command into a file, but instead appends to the file.\\

\columnbreak 
\textbf{Bash (Bourne-again Shell)}\\
Bash is a shell for the GNU operating system. Its features
include tab completion and unlimited command history,
accessible by the up and down arrows. To get into the
bash shell, simply type \verb+bash+ at the command prompt. To
get out of bash shell, simply type \verb+exit+.\\

\vspace{2mm}
\textbf{Basic Unix Commands}\\

\begin{tabular}{ l l }
bash & changes to bash shell\\
cat [files] & display or concatenate files\\
cat [file1] >> [file2] & append file1 to file2\\
cd & change directory\\
cp[file] [path] & copies a file to designated path\\
diff [file1] [file2] & compares two files, shows diff++\\
exit & exits out of shell\\
find [path] -name [file] & returns full path of file\\
g++ [file.cc] & compiles your c++ programs\\
grep [pattern] [file] & searches through files for pattern\\
head [file] & displays first 10 lines of file\\
history & displays command history\\
kill -9 [process id] & kills process by pid\\
less [file] & lets you scroll through output\\
lpr [file] & send file to printer\\
man [command] & reference page for command\\
mkdir [dir name] & creates a directory\\
passwd & changes password\\
ps & shows active processes\\
pwd & print working directory\\
rm [file] & remove file\\
rmdir [directory] & remove [empty] directory\\
tail [file] & displays last 10 lines of file\\
top & shows top 15 system processes\\
which [command] & shows full path of shell command\\
\end{tabular}

\vspace{2mm}
\textbf{Running files in the background}\\
When opening things from a terminal, you notice that if
you type in the command line, gedit, you will no
longer be able to run commands in your terminal. Instead,
type \verb+gedit &+ the ampersand '\verb+&+' tells it to run in the
background. Using this method, you can run multiple
programs and still retain the use of your command line.

\vspace{2mm}
\textbf{What is SFTP?}\\
Secure File Transfer Protocol is a protocol that uses SSH
and lets you manage, access, and transfer your files
securely over a network.

\vspace{2mm}
\textbf{SFTP Commands}\\
\begin{tabular}{ l l }
cd & change directory\\
dir/ls & display remote directory listing\\
exit/quit/bye & quit sftp\\
get [r-path] [l-path] & download file\\
help/? & display help text\\
lcd & change directory [locally]\\
lls & display local directory listing\\
lmkdir & make directory [locally]\\
lpwd & print local directory listing\\
mkdir & make directory [remotely]\\
put [l-path] [r-path] & upload file\\
pwd & print remote working directory\\
rename [old] [new] & rename remote file\\
rm & remove remote file\\
rmdir & remove remote directory\\
! & escape to local shell\\
![command] & execute command in local shell\\
\end{tabular}

\vspace{2mm}
\textbf{Using text editors}\\
Text editors are programs that are used to edit files. There
are many that you can use, such as: emacs, gedit, nano/pico
and vi/vim. To use these, simply type the editor name,
followed by the file name - \verb+emacs file.cc &+ which
will open an existing or create a new file with that name.

\vspace{2mm}
\textbf{Warning about closing terminal windows}\\
If you close a terminal window that you are currently using
to run a program (for instance, gedit) it will close down all
processes associated with that terminal window. Be careful
to remember what you used it for and save often!

\vspace{2mm}
References\\
"User Commands."\textit{Ubuntu 10.04.3 LTS, LucidLynx}, Man
Page Tomb, Date: Man Page Reference. Shell. Date
Accessed: January 29, 2012.

\end{multicols}
\end{landscape}
\end{document}
